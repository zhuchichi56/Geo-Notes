\documentclass{article}
\usepackage{ctex}  % 支持中文
\usepackage{amsmath}  % 数学公式
\usepackage{amsthm}  % 定理环境
\usepackage{amssymb}  % 数学符号
\usepackage{graphicx}  % 图片
\usepackage{enumitem}  % 列表
\usepackage[colorlinks=true,linkcolor=blue]{hyperref}  % 超链接

\title{时间序列分析:自相关系数学习笔记}
\author{统计学习笔记}
\date{\today}

\begin{document}

\maketitle

\section{基本概念}

自相关系数是时间序列分析中的基础概念,用于衡量时间序列在不同时间点之间的相关性。简单来说,它告诉我们"过去的值与未来的值有多相关"。

\subsection{定义与公式}

自相关系数的计算公式:

\begin{equation}
    \rho_\tau = \frac{\sum_{t=1}^{n-\tau}(x_t - \bar{x})(x_{t+\tau} - \bar{x})}{\sum_{t=1}^{n}(x_t - \bar{x})^2}
\end{equation}

其中:
\begin{itemize}
    \item $x_t$ 是时间t的观测值
    \item $\bar{x}$ 是序列的平均值
    \item $\tau$ 是时滞(time lag)
    \item n 是序列的总长度
\end{itemize}

\subsection{特点}
\begin{itemize}
    \item 取值范围:[-1, 1]
    \item $\tau = 0$ 时,自相关系数 = 1
    \item 一般建议时滞$\tau$取值不超过数据长度的1/4
\end{itemize}

\section{实际意义}

自相关系数帮助我们:
\begin{enumerate}
    \item 发现数据的周期性模式
    \item 评估数据的可预测性
    \item 识别时间序列中的规律
\end{enumerate}

例如:
\begin{itemize}
    \item $\tau=1$的高相关性:相邻时间点有很强的关联
    \item $\tau=7$的高相关性:可能存在周周期
    \item $\tau=12$的高相关性:月度数据可能存在年周期
\end{itemize}

\section{计算示例}

以5天温度数据为例:[20, 22, 21, 23, 19]°C

\subsection{步骤分解(计算$\tau=1$的自相关系数)}

\begin{enumerate}
    \item 计算平均值:
    \[ \bar{x} = (20 + 22 + 21 + 23 + 19) \div 5 = 21 \]
    
    \item 计算偏差值(与平均值的差):
    \begin{verbatim}
    20 - 21 = -1
    22 - 21 = 1
    21 - 21 = 0
    23 - 21 = 2
    19 - 21 = -2
    \end{verbatim}
    
    \item 计算分子(偏差乘积之和):
    \[ (-1)(1) + (1)(0) + (0)(2) + (2)(-2) = -5 \]
    
    \item 计算分母(偏差平方和):
    \[ (-1)^2 + (1)^2 + (0)^2 + (2)^2 + (-2)^2 = 10 \]
    
    \item 最终结果:
    \[ \rho_1 = -5/10 = -0.5 \]
\end{enumerate}

\subsection{矩阵表示}

自相关系数也可以用矩阵形式简洁地表示:

\begin{equation}
    \rho_\tau = \frac{\mathbf{X}^T\mathbf{X_\tau}}{\mathbf{X}^T\mathbf{X}}
\end{equation}

其中:
\[ 
\mathbf{X} = \begin{bmatrix} 
x_1 - \bar{x} \\
x_2 - \bar{x} \\
\vdots \\
x_{n-\tau} - \bar{x}
\end{bmatrix}, \quad
\mathbf{X_\tau} = \begin{bmatrix}
x_{1+\tau} - \bar{x} \\
x_{2+\tau} - \bar{x} \\
\vdots \\
x_n - \bar{x}
\end{bmatrix}
\]

\section{设计原理}

自相关系数公式的设计考虑了以下几个关键点:

\begin{enumerate}
    \item \textbf{去中心化}(减去平均值):
    \begin{itemize}
        \item 消除数据整体水平的影响
        \item 突出变化模式
    \end{itemize}
    
    \item \textbf{标准化}(除以方差):
    \begin{itemize}
        \item 使结果限制在[-1,1]区间
        \item 便于不同数据集之间的比较
    \end{itemize}
    
    \item \textbf{相乘项}:
    \begin{itemize}
        \item 正正/负负相乘为正,表示正相关
        \item 正负相乘为负,表示负相关
    \end{itemize}
    
    \item \textbf{时滞$\tau$}:
    \begin{itemize}
        \item 允许探测不同时间尺度的关系
        \item 有助于发现各种周期性模式
    \end{itemize}
\end{enumerate}

\section{应用示例}

商店销售额分析:
\begin{itemize}
    \item $\tau=1$的高相关:今天生意好,明天可能也好
    \item $\tau=7$的高相关:每周相同日子的销售模式相似
    \item 应用:人力资源调配、库存管理优化
\end{itemize}

\section{注意事项}

\begin{enumerate}
    \item 样本量要求:
    \begin{itemize}
        \item 计算高阶时滞需要足够多的数据
        \item 一般建议$\tau$不超过数据长度的1/4
    \end{itemize}
    
    \item 解释限制:
    \begin{itemize}
        \item 仅反映线性关系
        \item 需要与其他分析方法配合使用
    \end{itemize}
    
    \item 实际应用:
    \begin{itemize}
        \item 通常使用软件(如Python、R)计算
        \item 要注意数据的平稳性假设
    \end{itemize}
\end{enumerate}

\end{document}




