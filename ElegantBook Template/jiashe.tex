\documentclass[12pt]{article}
\usepackage{amsmath}
\usepackage{amssymb}
\usepackage{geometry}
\usepackage{enumitem}


\geometry{a4paper, margin=1in}

\title{通俗易懂的 t 检验和显著性判断}
\author{}
\date{}

\begin{document}

\maketitle

\section*{1. 什么是 t 检验?}
t 检验是一种统计方法,用于比较两组数据的平均值是否存在显著差异。常见于以下场景:
\begin{itemize}
    \item 药物实验:比较实验组(服药)和对照组(未服药)的健康指标差异。
    \item 教育研究:比较两种教学方法对学生成绩的影响。
\end{itemize}

\textbf{核心思想:}  
通过计算 t 值和 p 值,判断是否拒绝零假设 \( H_0 \) (即两组均值相等)。

---

\section*{2. t 检验公式}
t 值的计算公式如下:
\[
t = \frac{\bar{X}_1 - \bar{X}_2}{\sqrt{S_p^2 \cdot \left( \frac{1}{n_1} + \frac{1}{n_2} \right)}}
\]
其中:
\begin{itemize}
    \item \( \bar{X}_1, \bar{X}_2 \):实验组和对照组的样本均值。
    \item \( S_p^2 \):两组数据的合并方差:
    \[
    S_p^2 = \frac{(n_1 - 1) S_1^2 + (n_2 - 1) S_2^2}{n_1 + n_2 - 2}
    \]
    \item \( n_1, n_2 \):两组样本的大小。
    \item \( S_1, S_2 \):两组数据的标准差。
\end{itemize}

---

\section*{3. 显著性和 p 值}
\begin{itemize}
    \item \textbf{显著性水平(\(\alpha\)}:}  
    通常设定为 \( \alpha = 0.05 \),即允许 5\% 的概率犯第一类错误(错误拒绝一个正确的假设)。
    \item \textbf{p 值:}  
    \( p \) 值表示在零假设成立时,观察到当前数据或更极端结果的概率。
    \item \textbf{判断规则:}
    \begin{enumerate}[label=\arabic*.]
        \item 如果 \( p < \alpha \),拒绝零假设 \( H_0 \),结果具有统计显著性。
        \item 如果 \( p \geq \alpha \),无法拒绝零假设,结果不具有统计显著性。
    \end{enumerate}
\end{itemize}

---

\section*{4. 核心总结}
\begin{itemize}
    \item t 检验的目标是判断两组均值是否有显著差异。
    \item \textbf{规则:}  
    只要算出来的 \( p \) 值小于显著性水平 \( \alpha \)(比如 0.05),就可以拒绝零假设 \( H_0 \)。
    \item \( p \) 值越小,越有理由认为两组均值确实不同。
    \item \textbf{通俗理解:}  
    \( p \) 值是对零假设发出的“报警信号”,越小代表越不支持零假设,差异越显著。
\end{itemize}

---

\section*{5. 举例:药物实验}
假设:我们想测试一种药物是否有效。
\begin{itemize}
    \item \textbf{零假设(\( H_0 \)}:}药物无效(实验组和对照组的均值相等)。
    \item \textbf{备选假设(\( H_1 \)}:}药物有效(实验组和对照组的均值不同)。
\end{itemize}

实验结果:
\[
\bar{X}_1 = 120, \quad \bar{X}_2 = 125, \quad S_1 = 10, \quad S_2 = 12, \quad n_1 = n_2 = 30
\]

计算 \( t \) 值(略过计算细节):
\[
t = -1.75
\]

自由度 \( df = n_1 + n_2 - 2 = 58 \),查表得对应 \( p \) 值:
\[
p = 0.02
\]

\textbf{结论:}
\begin{itemize}
    \item \( p = 0.02 < \alpha = 0.05 \),因此拒绝零假设 \( H_0 \),说明药物对健康有显著效果。
\end{itemize}

---

\section*{6. 注意事项}
\begin{itemize}
    \item \( p \) 值越小,并不表示差异越大,而是表示结果越不可能是偶然发生的。
    \item 统计显著性(\( p \) 值小)不等于实际意义,需要结合效应大小和实际场景分析。
\end{itemize}

\end{document}


