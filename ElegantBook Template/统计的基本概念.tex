\documentclass{article}
\usepackage{CJKutf8}
\usepackage{amsmath}
\usepackage{geometry}
\usepackage{enumitem}

\geometry{a4paper,margin=2.5cm}

\begin{document}
\begin{CJK}{UTF8}{gbsn}

\title{统计推断与抽样理论重要概念总结}
\author{阅读笔记}
\date{\today}

\maketitle

\section{统计推断的基本概念}

\subsection{核心定义}
统计推断(Statistical inference)的本质是通过样本数据对总体特征进行推断。其中最重要的概念:
\begin{itemize}
    \item 总体(Population):研究对象的全体
    \item 样本(Sample):总体中的一部分个体
\end{itemize}

\subsection{重要结论}

\textbf{关于样本量的关键结论:}
\begin{itemize}
    \item 理论上样本越大越接近总体,结论越可靠
    \item 实践中1200-1500是比较理想的选择
    \item 社会调查中样本量不应低于1000
\end{itemize}

\section{抽样误差与偏性}

\subsection{两类基本误差}
\begin{enumerate}
    \item \textbf{偏差(Bias)}:统计量持续朝同一方向偏离总体参数值
    \item \textbf{变异性(Variability)}:不同样本间的结果差异程度
\end{enumerate}

\subsection{常见偏性类型}
\begin{itemize}
    \item 不回答偏性(Non-response bias):特定群体倾向于不回答
    \item 选择偏性(Selection bias):调查偏向易接触群体
\end{itemize}

\section{置信陈述}

置信陈述(Confidence Statement)是科学的总体推断表达方式,包含:
\begin{itemize}
    \item 误差界限(Margin of Error):样本统计量与总体参数的距离
    \item 置信水平(Level of Confidence):满足误差界限的样本比例
\end{itemize}

\textbf{规范表述示例:}
"经过调查,我们有95\%的把握相信,大约有29\%~31\%的总体具有该特征。"

\section{基本统计测度}

\subsection{均值(Mean)}
\[ \bar{x} = \frac{1}{n}\sum_{i=1}^n x_i \]
几何意义:表示数据的重心位置

\subsection{方差与标准差}
总体方差:
\[ \sigma^2 = \frac{1}{n}\sum_{i=1}^n (x_i - \bar{x})^2 \]

样本方差:
\[ S^2 = \frac{1}{n-1}\sum_{i=1}^n (x_i - \bar{x})^2 \]

\textbf{重要结论:}当样本容量足够大时,总体方差与样本方差的数值差别可以忽略。

\subsection{协方差}
\[ cov(x,y) = \frac{1}{n}\sum_{i=1}^n (x_i - \bar{x})(y_i - \bar{y}) \]

特殊情况:当$x_i = y_i$时,协方差退化为方差。

\end{CJK}
\end{document}


