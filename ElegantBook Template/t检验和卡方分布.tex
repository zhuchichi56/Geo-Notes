\section{t检验与卡方分布学习笔记}

\subsection{t检验基础}

\subsubsection{定义}
t检验用于比较两组数据的均值差异是否显著。

\subsubsection{计算公式}
\[t = \frac{\bar{X}_1 - \bar{X}_2}{S_p\sqrt{\frac{1}{n_1}+\frac{1}{n_2}}}\]

其中:
\begin{itemize}
    \item $\bar{X}_1, \bar{X}_2$ 是两组的平均值
    \item $S_p$ 是合并标准差
    \item $n_1, n_2$ 是两组的样本量
\end{itemize}

\subsubsection{合并标准差计算}
\[S_p = \sqrt{\frac{(n_1-1)S_1^2 + (n_2-1)S_2^2}{n_1+n_2-2}}\]

\subsubsection{实例分析}
比较两个班级的考试成绩:
\begin{itemize}
    \item A班:20个学生,平均分80分,标准差10分
    \item B班:20个学生,平均分85分,标准差10分
\end{itemize}

\paragraph{计算步骤}
\begin{enumerate}
    \item 计算合并标准差 $S_p = 10$
    \item 计算t值 = 1.58
    \item 自由度 = 38
    \item 查表得临界值($\alpha=0.05$)= 2.02
    \item 结论:$|1.58| < 2.02$,差异不显著
\end{enumerate}

\subsubsection{自由度解释}
\begin{itemize}
    \item 自由度 = 总样本数 - 被占用的自由度数
    \item 例:40个数据 - 2个平均值 = 38个自由度
\end{itemize}

\subsection{卡方分布}

\subsubsection{定义}
卡方分布用于检验实际观察值与理论期望值的差异。

\subsubsection{基本公式}
\[\chi^2 = \sum \frac{(实际值 - 理论值)^2}{理论值}\]

\subsubsection{实例:骰子公平性检验}
投掷120次骰子的理论与实际结果:

\begin{tabular}{|c|c|c|}
\hline
点数 & 理论期望 & 实际结果 \\
\hline
1 & 20 & 25 \\
2 & 20 & 18 \\
3 & 20 & 15 \\
4 & 20 & 22 \\
5 & 20 & 16 \\
6 & 20 & 24 \\
\hline
\end{tabular}

\subsection{显著性水平与临界值}

\subsubsection{显著性水平($\alpha$)}
\begin{itemize}
    \item 通常取0.05(95\%置信度)
    \item 表示允许5\%的误判风险
\end{itemize}

\subsubsection{临界值的意义}
\begin{itemize}
    \item 超过临界值:差异显著,可能不是偶然
    \item 未超过临界值:差异不显著,可能是随机波动
\end{itemize}

\subsubsection{决策规则}
对于t检验:
\[
\begin{cases}
拒绝原假设, & \text{if } |t| > t_\alpha \\
不拒绝原假设, & \text{if } |t| \leq t_\alpha
\end{cases}
\]

对于卡方检验:
\[
\begin{cases}
拒绝原假设, & \text{if } \chi^2 > \chi_\alpha^2 \\
不拒绝原假设, & \text{if } \chi^2 \leq \chi_\alpha^2
\end{cases}
\]

\subsection{关键区别}

\begin{tabular}{|c|c|c|}
\hline
特征 & t检验 & 卡方检验 \\
\hline
用途 & 比较两组均值差异 & 检验实际值与理论值的差异 \\
实例 & 比较两个班级成绩 & 检验骰子是否公平 \\
检验重点 & 均值差异 & 分布差异 \\
\hline
\end{tabular}

