\documentclass[12pt]{article}
\usepackage{amsmath}
\usepackage{geometry}
\usepackage{enumitem}
\geometry{a4paper, margin=1in}

\title{时间序列与随机过程核心概念笔记}
\author{}
\date{}

\begin{document}

\maketitle

\section*{1. 时间序列的组成与特征}
\begin{itemize}
    \item 时间序列是随时间变化的变量记录,广泛存在于地理、经济等领域。
    \item 实际中的时间序列通常是\textbf{多种成分叠加}的结果,包括:
    \begin{enumerate}[label=(\alph*)]
        \item \textbf{确定性成分}:如周期性变化(例如地球—太阳—月球的引力作用)。
        \item \textbf{随机性成分}:如噪声和不可预测的干扰。
    \end{enumerate}
    \item 公式表示:\[
    Y = T + C + I \quad \text{或} \quad Y = T \cdot C \cdot I
    \]
    其中:
    \begin{itemize}
        \item \( T \):趋势成分(增长或衰减)。
        \item \( C \):循环成分(如季节性波动)。
        \item \( I \):不规则随机成分。
    \end{itemize}
\end{itemize}

---

\section*{2. 时间序列的主要类型}
\begin{itemize}
    \item \textbf{准周期序列(Quasi-Periodic)}:近似周期,但周期不完全固定。例如潮汐受太阳和月球周期性影响,但其叠加关系不可通约。
    \item \textbf{暂态序列(Transient State)}:无明显周期性,但可能有趋势、跳跃或突变。例如:
    \begin{itemize}
        \item \textbf{跳跃(Jump):}急剧变化,反映系统从一种状态快速过渡到另一种状态。
        \item \textbf{突变(Catastrophe):}跳跃的特殊形式,系统可能会回归原状。
    \end{itemize}
    \item \textbf{趋势序列}:统计参数(如均值、方差等)呈现持续变化,通常由系统内部性质决定。
\end{itemize}

---

\section*{3. 随机过程的概念}
\begin{itemize}
    \item \textbf{随机过程(Stochastic Process)}:随时间推进的随机现象的数学抽象。
    \item 示例:
    \begin{itemize}
        \item 某地每年的降水量 \( x_t \)。
        \item 某公交车站不同时间的候车人数。
    \end{itemize}
    \item 一个随机过程可以用\textbf{随机时间序列}表示,而其变量本质上是随机变量。
    \item \textbf{随机变量(Random Variable)}:描述随机现象各种可能结果的变量,例如某时刻公交车站的候车人数。
\end{itemize}

---

\section*{4. 平稳时间序列与相依性}
\begin{itemize}
    \item \textbf{平稳时间序列}:统计特性(如均值、方差、自相关)不随时间变化的时间序列。
    \item 分类:
    \begin{enumerate}[label=(\alph*)]
        \item \textbf{独立随机序列:}序列中每个变量相互独立,无关联性。
        \item \textbf{相依随机序列:}序列变量之间有线性相关性,用\textbf{自相关系数}量化。
    \end{enumerate}
    \item \textbf{线性平稳随机模型}:
    \begin{itemize}
        \item 自回归模型(AR)。
        \item 自回归-移动平均模型(ARMA)。
    \end{itemize}
    \item 实例:空间分析中的\textbf{空间自相关分析},如 Moran’s I。
\end{itemize}

---

\section*{5. 时间序列分析方法}
\begin{itemize}
    \item \textbf{分解方法:}对于具有 \( Y = T + C + I \) 结构的时间序列,可通过分解提取趋势和周期成分。
    \item \textbf{数学变换:}如傅里叶分析,用于分解多周期叠加的复杂时间序列。
    \item \textbf{随机成分识别:}
    \begin{itemize}
        \item 通过绘制时间序列图(横轴为时间 \( t \),纵轴为变量值 \( x(t) \))初步识别类型。
        \item 借助统计量如\textbf{自相关系数}和卡方检验更准确地判断特征。
    \end{itemize}
\end{itemize}

---

\section*{6. 核心总结}
\begin{enumerate}[label=\arabic*.]
    \item 时间序列是多种成分(趋势、周期、随机等)叠加的结果,分析时需要分解和识别。
    \item \textbf{随机过程}可以用随机序列表示,其变量是随机变量。随机序列分为独立和相依两种类型。
    \item 分析工具:
    \begin{itemize}
        \item \textbf{简单方法:}时间序列图直观法。
        \item \textbf{统计方法:}自相关分析、随机模型(如 AR, ARMA)。
    \end{itemize}
    \item 重要应用:
    \begin{itemize}
        \item 识别跳跃(急剧变化)和突变(临界变化)。
        \item 研究复杂系统的时间序列行为,如气候变化、潮汐、社会经济系统。
    \end{itemize}
\end{enumerate}

---

\section*{7. 示例:降水量与时间序列}
\begin{itemize}
    \item 深圳市年降水量 \( x_t \) 是一个随机过程。
    \item 它受多种周期性(如季节)和随机性(如天气波动)因素的影响。
    \item 我们可以通过时间序列分析,提取其趋势(如年降水量增加或减少)、周期性(如雨季模式)和随机成分(短期波动)。
\end{itemize}

---

\end{document}

